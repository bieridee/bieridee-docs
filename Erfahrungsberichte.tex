\documentclass[10pt,a4paper]{scrartcl}
\pagestyle{empty}
\usepackage{a4} % alternativ \usepackage{a4wide}
\usepackage[ngerman]{babel} % Neudeutsche Silbentrennung (mehrsprachiges Dokument)
\usepackage{parskip} % Skip indentation of first row
\usepackage{graphicx} % Graphics support
\usepackage{longtable} % Tables across several pages
\usepackage{booktabs}
\usepackage{hyperref} % Hyperlinks
\usepackage[automark]{scrpage2} %kopf/fusszeile
\usepackage[utf8x]{inputenc} % Unicode-Encoding
 
\linespread{1.3}

\author{Danilo Bargen, Christian Fässler, Jonas Furrer} 
\title{Persönliche Erfahrungsberichte\\ Projekt BierIdee}

\pagestyle{scrheadings}
\ihead{SE2 Projekte} %linke Kopfzeile
\ohead{BierIdee} %rechte Kopfzeile

\begin{document}

\begin{titlepage}
	\maketitle
	\vspace{120mm}
	\thispagestyle{empty} % Don't start page numbers on this page
\end{titlepage}

\tableofcontents
\newpage

\section{Einführung}

\subsection{Zweck}
In diesem Dokument sind die persönlichen Erfahrungsberichte der Projektmitarbeiter des Projektes BierIdee.

\subsection{Gültigkeitsbereich}
Die Gültigkeit des Dokumentes beschränkt sich auf die Dauer des SE2-Projekte Modules FS2012.

\newpage
\section{Erfahrungsberichte}
\subsection{Danilo Bargen}
Es war cool!


\newpage
\subsection{Christian Fässler}
\subsubsection*{Beginn des Projektes}
Ich wusste vor dem Projekt nicht genau was mich erwarten würde, auch der erwartete Umfang war mir nicht klar. In der ersten Woche wurde das ganze dann aber ziemlich schnell diskret. Das Team hatte ich zum Glück schon bereits zum vorangegangenen Semesterende gefunden. Die Suche nach einer Projektidee gestaltete sich extrem schwierig, sodass wir dann relativ eng vor der Eingabe der Projektidee uns noch umentschieden.
\subsubsection*{Projektmanagement}
Ich bin beruflich zur Zeit nicht als Software Entwickler tätig und habe daher keine Erfahrung mit agiler Software Entwicklung. Sehr wohl aber in der Projektleitung von Infrastrukturprojekten. Schlussendlich muss ich sagen, dass ich daher im Bereich Projektmanagement in der Softwareentwicklung einiges dazulernen konnte. Im speziellen die iterative Evaluation der Requirements hat bei mir einen positiven Eindruck hinterlassen.

\subsubsection*{Programming}
Wir haben uns ein sehr sportliches Ziel gesetzt mit diesem Projekt. Es gab viele Frameworks und Technologien die wir erlernen mussten. Einerseits war die Androidentwicklung ein grösserer Teil davon - ein happiger Brocken, der mich aber sehr motiviert hat.
Schlussendlich muss ich sagen, hab ich sehr viel Know How vermittelt bekommen von meinen Kollegen oder im Selbststudium. Obwohl von neuen Technologien explizit abgeraten wurde.
Meine Ansicht dazu ist aber, dass es in IT Projekten NIE ohne "Neues" geht. Das ist sicher Einstellungssache, aber die neuen Technologien haben mich sehr angespornt. 

\subsubsection*{Teamarbeit}
Es war mein erstes grösseres Softwareprojekt, sprich in einem Team. Meine Kollegen haben da im Gegensatz zu mir bereits einiges an Erfahrung und konnten mich daher gut und schnell in die SW Teamarbeit instruieren (Continous Integration, GitHub etc.). Durch das kleine Projektteam konnten wir sehr unkompliziert über akute Probleme diskutieren und schnell Entscheidungen treffen. Wir verfolgten stets den Ansatz, dass jeder in jedem Bereich arbeiten soll, und hatten nicht von Anfang an Zuständigkeiten zugeteilt. Zu meinem Erstaunen hat das auch sehr gut funktioniert. Die Teamarbeit hat sehr gut funktioniert. Auch die Arbeitsverteilung war meiner Meinung nach ausgeglichen. 

\subsubsection*{Fazit}
Das Projekt habe ich seiner grösse Unterschätzt (siehe IST Aufwand). Zu unserer Verteidung muss ich aber auch sagen, dass unsere erste Projektidee einen geringeren Umfang gehabt hätte, dann aber abgelehnt wurde. Das Projekt stand im Hauptfokus in diesem Semester, daneben blieb für andere Fächer kaum mehr Zeit. Vielleicht sollte es für dieses Fach 6 Punkte geben. Insgesamt hat das Projekt Spass gemacht und ich bin um eine gute Erfahrung reicher.

\newpage
\subsection{Jonas Furrer}
Das Projekt BierIdee war in meinen Augen ein Erfolg. Zu diesem Erfolg gehören für mich verschiedene Aspekte, einer davon 
ist der Lerneffekt.\\
Obwohl ich neben dem Studium als Java-Entwickler arbeite konnte ich bei der Durchführung dieses Projektes sehr viel 
lernen. Einerseits hatte ich noch nie zuvor mit Android gearbeitet und andererseits ist die Durchführung und der Aufbau
eines Projektes von Grund auf bis zum fertigen Produkt etwas, das ich in dieser Form noch nicht gemacht hatte.\\
Ich konnte also nicht nur im Bereich der Technik sondern auch stark im Bereich der Projekt-Planung und der 
Architektur- und Design-Planug profitieren und dazulernen.\\
Ein weiterer Aspekt des Erfolges ist für mich auch die Teamarbeit. Jedes Teammitglied kommt aus einem anderen 
Arbeitsumfeld und bringt somit auch einen anderen Erfahrungsschatz mit. Wir konnten uns Dank dieser Tatsache gegenseitig
in verschiedenen Bereichen unterstützen und neue Ideen in das Projekt einfliessen lassen. Dank der eher geringen
Teamgrösse von drei Personen, konnten wir das Know-How der Verschiedenen Projektbereiche sehr gut verteilen. Daduch
hatte man immer einen Diskusionspartner, der einem bei Problemen weiterhelfen konnte. Ich habe die Zusammenarbeit mit 
meinen Teamkollegen sehr geschätzt und habe die Arbeitsverteilung als sehr ausgewogen empfunden.\\
Ich war erstaunt wie viel man wir in dieser relativ kurzen Zeit erreichen konnten, und andrereseits erstaunte mich auch
wieviel Zeit gewisse Aufgaben in Anspruch nehmen können. Insgesammt war das Projekt sehr Zeitaufwändig, das Projekt
prägte gewissermassen dieses Semester. Die meiste Lernzeit und auch nicht wenig von meiner Freizeit habe ich in dieses
Projekt investiert. Ich habe den Eindruck, dass ich entsprechend meinen Investitionen profitieren konnte.


\end{document}
