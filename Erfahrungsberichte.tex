\documentclass[10pt,a4paper]{scrartcl}
\pagestyle{empty}
\usepackage{a4} % alternativ \usepackage{a4wide}
\usepackage[ngerman]{babel} % Neudeutsche Silbentrennung (mehrsprachiges Dokument)
\usepackage{parskip} % Skip indentation of first row
\usepackage{graphicx} % Graphics support
\usepackage{longtable} % Tables across several pages
\usepackage{booktabs}
\usepackage{hyperref} % Hyperlinks
\usepackage[automark]{scrpage2} %kopf/fusszeile
\usepackage[utf8x]{inputenc} % Unicode-Encoding
 
\linespread{1.3}

\author{Danilo Bargen, Christian Fässler, Jonas Furrer} 
\title{Persönliche Erfahrungsberichte\\ Projekt BierIdee}

\pagestyle{scrheadings}
\ihead{SE2 Projekte} %linke Kopfzeile
\ohead{BierIdee} %rechte Kopfzeile

\begin{document}

\begin{titlepage}
	\maketitle
	\vspace{120mm}
	\thispagestyle{empty} % Don't start page numbers on this page
\end{titlepage}

\tableofcontents
\newpage

\section{Einführung}

\subsection{Zweck}
In diesem Dokument sind die persönlichen Erfahrungsberichte der Projektmitarbeiter des Projektes BierIdee.

\subsection{Gültigkeitsbereich}
Die Gültigkeit des Dokumentes beschränkt sich auf die Dauer des SE2-Projekte Modules FS2012.

\newpage
\section{Erfahrungsberichte}
\subsection{Danilo Bargen}
Es war cool!


\newpage
\subsection{Christian Fässler}
Es hat Spass gemacht!


\newpage
\subsection{Jonas Furrer}
Das Projekt BierIdee war in meinen Augen ein Erfolg. Zu diesem Erfolg gehören für mich verschiedene Aspekte, einer davon 
ist der Lerneffekt.\\
Obwohl ich neben dem Studium als Java-Entwickler arbeite konnte ich bei der Durchführung dieses Projektes sehr viel 
lernen. Einerseits hatte ich noch nie zuvor mit Android gearbeitet und andererseits ist die Durchführung und der Aufbau
eines Projektes von Grund auf bis zum fertigen Produkt etwas, das ich in dieser Form noch nicht gemacht hatte.\\
Ich konnte also nicht nur im Bereich der Technik sondern auch stark im Bereich der Projekt-Planung und der 
Architektur- und Design-Planug profitieren und dazulernen.\\
Ein weiterer Aspekt des Erfolges ist für mich auch die Teamarbeit. Jedes Teammitglied kommt aus einem anderen 
Arbeitsumfeld und bringt somit auch einen anderen Erfahrungsschatz mit. Wir konnten uns Dank dieser Tatsache gegenseitig
in verschiedenen Bereichen unterstützen und neue Ideen in das Projekt einfliessen lassen. Dank der eher geringen
Teamgrösse von drei Personen, konnten wir das Know-How der Verschiedenen Projektbereiche sehr gut verteilen. Daduch
hatte man immer einen Diskusionspartner, der einem bei Problemen weiterhelfen konnte. Ich habe die Zusammenarbeit mit 
meinen Teamkollegen sehr geschätzt und habe die Arbeitsverteilung als sehr ausgewogen empfunden.\\
Ich war erstaunt wie viel man wir in dieser relativ kurzen Zeit erreichen konnten, und andrereseits erstaunte mich auch
wieviel Zeit gewisse Aufgaben in Anspruch nehmen können. Insgesammt war das Projekt sehr Zeitaufwändig, das Projekt
prägte gewissermassen dieses Semester. Die meiste Lernzeit und auch nicht wenig von meiner Freizeit habe ich in dieses
Projekt investiert. Ich habe den Eindruck, dass ich entsprechend meinen Investitionen profitieren konnte.


\end{document}
