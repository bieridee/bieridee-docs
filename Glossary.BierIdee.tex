\documentclass[10pt,a4paper]{scrartcl}
\pagestyle{empty}
\usepackage{a4} % alternativ \usepackage{a4wide}
\usepackage[utf8]{inputenc} % Unicode
\usepackage[ngerman]{babel} % Neudeutsche Silbentrennung (mehrsprachiges Dokument)
\usepackage{parskip} % Skip indentation of first row
\usepackage{graphicx} % Graphics support
\usepackage{longtable} % Tables across several pages
\usepackage{booktabs}
\usepackage{hyperref} % Hyperlinks
\usepackage[automark]{scrpage2} %kopf/fusszeile

\linespread{1.3}

\author{Danilo Bargen, Christian Fässler, Jonas Furrer} 
\title{Glossary}

\pagestyle{scrheadings}
\ihead{SE2 Projekte} %linke Kopfzeile
\ohead{BierIdee} %rechte Kopfzeile

\begin{document}

\begin{titlepage}
	\maketitle
	\vspace{120mm}
	\thispagestyle{empty} % Don't start page numbers on this page
\end{titlepage}

\section{Änderungshistorie}
\begin{tabular}{p{0.1\textwidth}p{0.15\textwidth}p{0.55\textwidth}p{0.1\textwidth}}
\toprule
\textbf{Version} & \textbf{Datum} & \textbf{Änderung} & \textbf{Person} \\  
\midrule
v1 & 06.03.2012 & Dokument erstellt, Domainmodel Sektion hinzugefügt & jfurrer \\  
\hline
v1.1 & 20.03.2012 & Architekturkomponenten hinzugefügt & dbargen \\  
\hline 
v1.2 & 01.04.2012 & Korrekturen nach Review & jfurrer \\
\bottomrule
\end{tabular} 
\newpage


\section{Glossary}

\subsection{Architekturkomponenten}
\begin{description}
	\item[Frontend] Alle Komponenten, die direkt vom Benutzer bedient werden, wie zB die Android App.
	\item[Backend] Alle Komponenten, die mit der Datenverarbeitung zu tun haben und die diese Daten dem Frontend bereitstellen.
	\item[Android App] Die Bieridee-App für Android, momentan unsere einzige Frontend-Komponente.
	\item[REST API] Die API, die dem Frontend-Teil die Daten via HTTP/JSON bereitstellt. Implementiert mit Restlet.
	\item[Datenbank] Die PostgreSQL Datenbank, welche alle relevanten Daten enthält und auf welche von der REST API aus (und nur von der REST API aus) zugegriffen wird.
\end{description}


\end{document}