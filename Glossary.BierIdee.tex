\documentclass[10pt,a4paper]{scrartcl}
\pagestyle{empty}
\usepackage{a4} % alternativ \usepackage{a4wide}
\usepackage[utf8]{inputenc} % Unicode
\usepackage[ngerman]{babel} % Neudeutsche Silbentrennung (mehrsprachiges Dokument)
\usepackage{parskip} % Skip indentation of first row
\usepackage{graphicx} % Graphics support
\usepackage{longtable} % Tables across several pages
\usepackage{hyperref} % Hyperlinks
\usepackage[automark]{scrpage2} %kopf/fusszeile

\linespread{1.3}

\author{Danilo Bargen, Christian Fässler, Jonas Furrer} 
\title{Glossary}

\pagestyle{scrheadings}
\ihead{SE2 Projekte} %linke Kopfzeile
\ohead{BierIdee} %rechte Kopfzeile

\begin{document}

\begin{titlepage}
	\maketitle
	\vspace{120mm}
	\thispagestyle{empty} % Don't start page numbers on this page
\end{titlepage}



\section{Glossary}
\subsection{Domainmodell}
\begin{description}
	\item[Bier] Dies bezeichnet ein Bier, welches sich über das Bierrezept definiert. Konkret beispielsweise eine Flasche Appenzeller Quöllfrisch oder eine Büchse Feldschlösschen Premium. Die Verpackung spielt dabei keine Rolle.
	\item[Bierkonsum] Eine konkrete konsumierte Quantität von Bier, z.B ein Glas Guinnes. Der Bierkonsum ist auch eine \texttt{Aktivität}.
	\item[Biersorte] Die grobe Einteilung eines Biers. Beispiele für Sorten sind, 'Weissbier', 'Weizenbier', 'Zwickelbier' und so weiter.
	\item[Bewertung] Die Bewertung ist die subjektive Einordnung der güte eine Bieres innerhalb einer definierten Skala durch einen Benutzer.
	\item[Empfehlung] Eine durch das system errechnete Bier-Empfehlung an einen Benutzer. Die Empfehlung basiert auf Vergleichen von vorlieben verschiedener Benutzer.
	\item[Tag] Mit Hilfe von Tags können Biere mit Schlagworten beschrieben werden.
	\item[Profil] Mit Profil ist explizit ein Brauerei-Profil gemeint. Das Profil beinhaltet eine undefinierte Menge an Informationen zu einer Brauerei.
	\item[Aktiviäten] Als Aktivitäten werden einerseits das Bewerten eines Bieres sowie die manuell getrackte Konsumation eines Bieres bezeichnet. 
	\item[Timeline] Bei der Timeline handelt es sich um eine chonologische liste aller Aktivitäten die vom System getrack werden.
	\item[Grösse] Die Grösse teilt die Brauereien in eine definierte Grössenkategorie ein. Die Grösse ist nur für Brauereien bestimmt. Grössenwerte könnten sein, 'mikro', 'regional', 'national', 'international'.
\end{description}




\end{document}