\documentclass[10pt,a4paper]{scrartcl}
\pagestyle{empty}
\usepackage{a4} % alternativ \usepackage{a4wide}
\usepackage[ngerman]{babel} % Neudeutsche Silbentrennung (mehrsprachiges Dokument)
\usepackage{parskip} % Skip indentation of first row
\usepackage{graphicx} % Graphics support
\usepackage{longtable} % Tables across several pages
\usepackage{booktabs}
\usepackage{hyperref} % Hyperlinks
\usepackage{float} % Force float position
\usepackage[automark]{scrpage2} %kopf/fusszeile
\usepackage{listings}
\usepackage[utf8x]{inputenc} % Unicode-Encoding

\lstset{basicstyle=\ttfamily}

\linespread{1.3}

\author{Danilo Bargen, Christian Fässler, Jonas Furrer} 
\title{REST Specfications\\Projekt BierIdee}

\pagestyle{scrheadings}
\ihead{SE2 Projekte} %linke Kopfzeile
\ohead{BierIdee} %rechte Kopfzeile

\let\textquotedbl="

\begin{document}

\begin{titlepage}
	\maketitle
	\vspace{120mm}
	\thispagestyle{empty} % Don't start page numbers on this page
\end{titlepage}

\newpage
	\tableofcontents
\newpage

\section*{Änderungshistorie}
\begin{tabular}{p{0.1\textwidth}p{0.15\textwidth}p{0.55\textwidth}p{0.1\textwidth}}
\toprule
\textbf{Version} & \textbf{Datum} & \textbf{Änderung} & \textbf{Person} \\  
\midrule
v1.0 & 03.04.2012 & Dokument erstellt & dbargen \\  
\hline 
v1.1 & 05.04.2012 & Tags und JSON Formate hinzugefügt & jfurrer \\
\hline 
v1.2 & 06.04.2012 & Überarbeitet, BeerType hinzugefügt & dbargen \\
\hline 
v1.3 & 24.04.2012 & Tagresource Beers entfernt, Paging bei BeerList hinzugefügt & jfurrer \\
\hline
v1.4 & 03.05.2012 & Definitionen von Consumptions und Ratings Resourcen angepasst & cfaessle\\
\hline
v1.5 & 12.05.2012 & Definitionen von Tag und Timeline Resourcen angepasst & jfurrer\\
\bottomrule
\end{tabular} 
\newpage


\section{Einleitung}

Die Definition der Ressourcen orientiert sich an den Regeln des Buches
\textit{REST API Design Rulebook} \cite{masse2011rest} aus dem O'Reilly Verlag.

\subsubsection*{URI Definition}

Bei der Bezeichnung der URIs\footnote{Uniform Resource Identifier} wurde folgende Terminologie gemäss RFC 3986 verwendet:

\texttt{URI = scheme \textquotedbl ://\textquotedbl{} authority \textquotedbl /\textquotedbl{}
path [ \textquotedbl ?\textquotedbl{} query ] [ \textquotedbl \#\textquotedbl{} fragment ]}

\subsubsection*{Ressource-Archetypen}

Nachfolgend die Ressource-Archetypen gemäss \cite{masse2011rest}. Die Erklärungstexte wurden direkt dem besagten
Buch entnommen.

\begin{description}
	\item[Document] A document resource is a singular concept that is akin to an object instance or database
		record. A document’s state representation typically includes both fields with values and
		links to other related resources.
	\item[Collection] A collection resource is a server-managed directory of resources. Clients may propose
		new resources to be added to a collection. However, it is up to the collection to choose
		to create a new resource, or not.
	\item[Store] A store is a client-managed resource repository. A store resource lets an API client put
		resources in, get them back out, and decide when to delete them. On their own, stores
		do not create new resources; therefore a store never generates new URIs. Instead, each
		stored resource has a URI that was chosen by a client when it was initially put into the
		store.
	\item[Controller] A controller resource models a procedural concept. Controller resources are like
		executable functions, with parameters and return values; inputs and outputs.
		Like a traditional web application’s use of HTML forms, a REST API relies on controller
		resources to perform application-specific actions that cannot be logically mapped to
		one of the standard methods (create, retrieve, update, and delete, also known as
		CRUD).
\end{description}


\section{Repräsentations-Designprinzipien}

Die JSON-Repräsentation ist abhängig vom Ressource-Archetypen.

\begin{description}
	\item[Document] Ein Document gibt ein JSON Objekt zurück, welches alle relevanten Felder enthält.
		Informationen, welche durch die Ressource-URI bereits gegeben sind (zB type), müssen nicht
		erneut in der Liste erscheinen.\\
		Falls im Document Unterobjekte auftauchen (zB das Bier einer Bewertung), wird für das Feld ein
		JSON-Objekt erstellt, welches die Ressource-URI und falls sinnvoll ein oder zwei relevante
		Felder enthält.\\
		Beispiel: \hfill
\begin{lstlisting}
{
	feld1: "{wert1}",
	feld2: "{wert2}",
	unterobjekt: {
		feld1: "{wert1}",
		uri: "{ressource-uri}"
	},
	untercollection: [
		{
			feld1: "{wert1}",
			uri: "{ressource-uri}"
		},
		{
			feld1: "{wert1}",
			uri: "{ressource-uri}"
		}
	]
}
\end{lstlisting}
	\item[Collection] Eine Collection gibt eine Liste mit allen enthaltenen (ggf. gefilterten)
		JSON-Objekten zurück. Die Objekte werden zusätzlich jeweils um ein Feld \texttt{uri} ergänzt,
		welches die Ressource-URI des jeweiligen Objektes enthält.\\
		Beispiel: \hfill
\begin{lstlisting}
{
	{
		feld1: "{wert1}",
		uri: "{ressource-uri}"
	},
	{
		feld1: "{wert1}",
		uri: "{ressource-uri}"
	},
}
\end{lstlisting}
	\item[Store] Ein Store verhält sich wie eine Collection, wenn sie direkt angesprochen wird
		(\texttt{/store}) und wie ein Document, wenn ein spezifisches Element des Store angesprochen
		wird (\texttt{/store/\{element-id\}}).
	\item[Controller] Der Output des Controllers ist abhängig vom Verwendungszweck.
\end{description}


\section{REST Ressourcen}

Nachfolgend sind die verfügbaren REST Ressourcen definiert. Alle Ressourcen sind
unter der URI Authority \texttt{http://brauhaus.nusszipfel.com/} erreichbar.


\subsection{Beer}

Ein spezifisches Bier, identifiziert durch die ID.

\begin{description}
	\item[URI Path] \texttt{/beers/\{beer-id\}}
	\item[Archetype] Document
	\item[Methods] GET, PUT, DELETE
	\item[JSON Format] \hfill
\begin{lstlisting}
{	
	id: "{beer-id}",
	name: "{beer-name}",
	image: "{image-path}",
	brand: "{brand-name}",
	beertype: "{ressource-URI}",
	tags: [{
		name: "{tag-name}",
		uri: "{ressource-URI}"
	}]
}
\end{lstlisting}
\end{description}


\subsection{Beers}

Der Bestand aller Biere.

\begin{description}
	\item[URI Path] \texttt{/beers}
 	\item[Query Parameters] \texttt{tag=\{tag-name\}, pageSize=\{size\}, pageStartIndex=\{index\}, user=\{username\}}
	\item[Archetype] Collection
	\item[Methods] GET, POST
	\item[JSON Format] \hfill
\begin{lstlisting}
[<beer document + uri>, ...]
\end{lstlisting}
\end{description}


\subsection{Beertype}

Ein Biertyp, identifiziert durch die ID.

\begin{description}
	\item[URI Path] \texttt{/beertypes/\{beertype-id\}}
	\item[Archetype] Document
	\item[Methods] GET, PUT, DELETE
	\item[JSON Format] \hfill
\begin{lstlisting}
{	
	id: "{beertype-id}",
	name: "{beertype-name}",
	description: "{description}",
	uri : "{Ressource-URI}"
}
\end{lstlisting}
\end{description}


\subsection{Beertypes}

Der Bestand aller Biertypen.

\begin{description}
	\item[URI Path] \texttt{/beertypes}
	\item[Archetype] Collection
	\item[Methods] GET, POST
	\item[JSON Format] \hfill
\begin{lstlisting}
[<beertype document + uri>, ...]
\end{lstlisting}
\end{description}

\subsection{Usercredentials}

Eine Controller-Ressource, welche einen signierten POST-Request entgegennimmt und die
HMAC-Authorisierungsdetails überprüft.

\begin{description}
	\item[URI Path] \texttt{/usercredentials}
	\item[Archetype] Controller
	\item[Methods] POST
	\item[Response Statuscodes] \hfill
		\begin{description}
			\item[204 No Content] Logindaten sind gültig
			\item[401 Unauthorized] Logindaten sind ungültig
		\end{description}
\end{description}


\subsection{User}

Ein Benutzer, identifiziert durch den Benutzernamen.

Um einen User zu erstellen, einen PUT Request ohne Authorization Header absenden. Der Username muss
dabei nicht mitgesendet werden, da er schon in der URI enthalten ist.

Um einen User zu ändern, einen authorisierten PUT Request absenden.

\begin{description}
	\item[URI Path] \texttt{/users/\{username\}}
	\item[Archetype] Document
	\item[Methods] GET, PUT, DELETE
	\item[Response Statuscodes] \hfill
		\begin{description}
			\item[201 Created] User wurde erfolgreich erstellt
			\item[204 No Content] User wurde erfolgreich geändert
			\item[400 Bad Request] Response Entity fehlt oder ist unvollständig
			\item[401 Unauthorized] Authorisierungsdaten sind ungültig oder fehlen
			\item[404 Not Found] Zu ändernder User nicht gefunden
			\item[409 Conflict] Zu erstellender User existiert bereits
		\end{description}
	\item[JSON Format] \hfill
\begin{lstlisting}
{
	username: "{username}",
	firstName: "{first-name}",
	lastName: "{last-name}",
	email: "{email}"
}
\end{lstlisting}
\end{description}


\subsection{Users}

Der Bestand aller User.

\begin{description}
	\item[URI Path] \texttt{/users}
	\item[Archetype] Collection
	\item[Methods] GET, POST
	\item[JSON Format] \hfill
\begin{lstlisting}
[<user document + uri>, ...]
\end{lstlisting}
\end{description}


\subsection{Recommendations}

Bier-Empfehlungen für einen bestimmten Benutzer.

\begin{description}
	\item[URI Path] \texttt{/users/\{username\}/recommendations}
	\item[Archetype] Controller
	\item[Methods] GET
	\item[JSON Format] \hfill
\begin{lstlisting}
[{
	"id": "{beer-id}",
	"name": "{beer-name}",
	"uri": "{ressource-URI}"
}]
\end{lstlisting}
\end{description}


\subsection{Rating}

Die letzte Bier-Bewertung durch einen bestimmten Benutzer. Eine neue Bewertung kann mittels \texttt{POST} hinzugefügt werden.

\begin{description}
	\item[URI Path] \texttt{/beers/\{beer-id\}/ratings/\{username\}}
	\item[Archetype] Controller
	\item[Methods] GET, POST
	\item[JSON Format] \hfill
\begin{lstlisting}
{
	beer: {
		name: "{beer-name}",
		uri: "{ressource-URI}"
	},
	user: {
		username: "{username}",
		uri: "{ressource-URI)"
	},
	value: "{value}"
}
	\end{lstlisting}
\end{description}

\subsection{Consumption}

Ein Bierkonsum, identifiziert durch die ID.

\begin{description}
	\item[URI Path] \texttt{/beers/\{beer-id\}/consumptions/\{consumption-id\}}
	\item[Archetype] Readonly Document
	\item[Methods] GET
	\item[JSON Format] \hfill
\begin{lstlisting}
{
	beer: {
		name: "{beer-name}",
		uri: "{ressource-URI}"
	},
	user: {
		username: "{username}",
		uri: "{ressource-URI)"
	},
	timestamp: "{value}"
}
\end{lstlisting}
\end{description}


\subsection{Consumptions}

Der Bestand aller Bierkonsume für ein bestimmtes Bier.

\begin{description}
	\item[URI Path] \texttt{/beers/\{beer-id\}/consumptions}
 	\item[Query Parameters] \texttt{user=\{username\}}
	\item[Archetype] Collection
	\item[Methods] GET, POST
	\item[JSON Format] \hfill
\begin{lstlisting}
[<consumption document + uri>, ...]
\end{lstlisting}
\end{description}


\subsection{Brewery}

Eine Brauerei, identifiziert durch die ID.

\begin{description}
	\item[URI Path] \texttt{/breweries/\{brewery-id\}}
	\item[Archetype] Document
	\item[Methods] GET, PUT, DELETE
	\item[JSON Format] \hfill
\begin{lstlisting}
{	
	id: "{brewery-id},
	name: "{brewery-name}",
	size: "{value}",
	description: "{brewery-description}",
	picture: "{brewery-picture}",
	uri : "{ressource-URI}"
}
\end{lstlisting}
\end{description}


\subsection{Breweries}

Der Bestand aller Brauereien.

\begin{description}
	\item[URI Path] \texttt{/breweries}
	\item[Query Parameters] \texttt{brewerySize=\{size\}}
	\item[Archetype] Collection
	\item[Methods] GET, POST
	\item[JSON Format] \hfill
\begin{lstlisting}
[<brewery document + uri>, ...]
\end{lstlisting}
\end{description}


\subsection{Timeline}

Die Aktivitäts-Timeline.

\begin{description}
 	\item[URI Path] \texttt{/timeline}
	\item[Query Parameters] \texttt{pageSize=\{size\}, pageStartIndex=\{index\}, user=\{username\}}
	\item[Archetype] Readonly Collection
	\item[Methods] GET
	\item[JSON Format] \hfill
\begin{lstlisting}
[{	
	type: "consumpton | rating",
	date: "{date}",
	timestamp: {timestamp},
	beer: {
		name: "{beer-name}",
		uri: "{beer-uri}"
	},
	user: {
		user: "{user-name}",
		uri: "{user-uri}"	
	},
	rating: {rating}
}]
\end{lstlisting}
\end{description}


\subsection{Tag}

Ein Tag identifizert durch die ID.

\begin{description}
	\item[URI Path] \texttt{/tags/\{tag-id\}}
	\item[Archetype] Document
	\item[Methods] GET, PUT, DELETE
	\item[JSON Format] \hfill
\begin{lstlisting}
[{	
	id: "{tag-id}"
	name: "{tag-name}",
	uri: "{ressource-URI}"
}]
\end{lstlisting}
\end{description}

\subsection{Tags}

Liste aller Tags.

\begin{description}
	\item[URI Path] \texttt{/tags}
	\item[Archetype] Store
	\item[Methods] GET, POST
	\item[JSON Format] \hfill
\begin{lstlisting}
[<tag document>, ...]
\end{lstlisting}
\end{description}


\bibliographystyle{alpha}
\bibliography{REST.Interface}

\end{document}
