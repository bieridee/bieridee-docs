\documentclass[10pt,a4paper]{scrartcl}
\pagestyle{empty}
\usepackage{a4} % alternativ \usepackage{a4wide}
\usepackage[ngerman]{babel} % Neudeutsche Silbentrennung (mehrsprachiges Dokument)
\usepackage{parskip} % Skip indentation of first row
\usepackage{graphicx} % Graphics support
\usepackage{longtable} % Tables across several pages
\usepackage{booktabs}
\usepackage{hyperref} % Hyperlinks
\usepackage{float} % Force float position
\usepackage[automark]{scrpage2} %kopf/fusszeile
\usepackage[utf8x]{inputenc} % Unicode-Encoding

% amsmath (large fractions)
\usepackage{amsmath}

% Tikz
\usepackage{tikz}
\usetikzlibrary{arrows}
 
\linespread{1.3}

\author{Danilo Bargen, Christian Fässler, Jonas Furrer} 
\title{Bier-Empfehlungen\\ Projekt BierIdee}

\pagestyle{scrheadings}
\ihead{SE2 Projekte} %linke Kopfzeile
\ohead{BierIdee} %rechte Kopfzeile

\begin{document}

\begin{titlepage}
	\maketitle
	\vspace{120mm}
	\thispagestyle{empty} % Don't start page numbers on this page
\end{titlepage}

\tableofcontents
\newpage

\section*{Änderungshistorie}
\begin{tabular}{p{0.1\textwidth}p{0.15\textwidth}p{0.55\textwidth}p{0.1\textwidth}}
\toprule
\textbf{Version} & \textbf{Datum} & \textbf{Änderung} & \textbf{Person} \\  
\midrule
v1.0 & 15.05.2012 & Dokument erstellt & dbargen \\  
\hline
v1.1 & 22.05.2012 & Dokument erweitert & dbargen \\  
\hline
v1.2 & 23.05.2012 & Dokument fertiggestellt & dbargen \\  
\bottomrule
\end{tabular} 
\newpage


\section{Einleitung}

Der Zweck dieses Dokumentes ist es, den im Projekt BierIdee verwendeten Bierempfehlungs-Algorithmus
zu erläutern. Der Algorithmus dient dazu, einem Benutzer ein Bier zu empfehlen, welches ihm
möglicherweise gut schmeckt.


\section{Glossar}

In diesem Dokument werden folgende Bezeichnungen verwendet:

\begin{description}
	\item[Zieluser] Der User, der sich eine Empfehlung ausstellen lassen will.
	\item[Like-Nachbarn] Andere User, deren Biergeschmack sich mit dem des Zielusers überschneidet.
	\item[Beliebtes Bier] Ein Bier, das eine Bewertung zwischen 3 und 5 erhalten hat.
\end{description}


\section{Verfahren}

Um die Empfehlungen zu ermitteln, werden verschiedene Faktoren berechnet und unterschiedlich
gewichtet. Schlussendlich werden diese Faktoren multipliziert, um eine Gesamtgewichtung für das
Bier zu erhalten. Die Biere mit der höchsten Gewichtung werden dann zu Empfehlungen für den
Zieluser.

\subsection{Schritt 1: Like-Nachbarn}

Um die Like-Nachbarn zu ermitteln, geht man von allen beliebten Bieren des Zielusers aus (d.h. von
ihm mit $\ge$ 3 bewertet). Alle User, die diese Biere auch mit 3 bis 5 Sternen bewertet haben, sind
Like-Nachbarn.

\begin{figure}[H]
	\begin{center}
		\begin{tikzpicture}[->,>=stealth',shorten >=1pt,auto,node distance=3cm,thick,scale=2.0,
				zieluser/.style={circle,draw,font=\sffamily\large\bfseries},
				beer/.style={circle,draw,font=\sffamily\large\bfseries},
				user/.style={circle,draw,font=\sffamily\large\bfseries,color=red,fill=red!10!orange!10},
			]

			\node[zieluser] (Z) at (3,3) {Zieluser};

			\node[beer] (B1) at (2,2) {Bier1};
			\node[beer] (B2) at (4,2) {Bier2};

			\node[user] (U1) at (1,1) {User1};
			\node[user] (U2) at (3,1) {User2};
			\node[user] (U3) at (5,1) {User3};

			\path[every node/.style={font=\sffamily\small}]
				(Z)
					edge node [swap] {rates 4} (B1)
					edge node {rates 5} (B2)
				(U1)
					edge node {rates 4} (B1)
				(U2)
					edge node {rates 5} (B1)
					edge node [swap] {rates 3} (B2)
				(U3)
					edge node [swap] {rates 5} (B2);

		\end{tikzpicture}
	\end{center}
	\caption{Beispiel Like-Nachbarn}
	\label{fig:beispiel-like-nachbarn}
\end{figure}

\newpage

Die Bewertungen werden dabei in fix definierten Stufen gewichtet:

\begin{table}[H]
	\begin{center}
		\begin{tabular}{cc}
			\toprule
			Bewertung & Gewichtung \\
			\midrule
			3 & 0.5 \\
			4 & 1 \\
			5 & 2 \\
			\bottomrule
		\end{tabular}
	\end{center}
	\caption{Bewertungs-Gewichtungen}
	\label{table:bewertungs-gewichtungen}
\end{table}

Für jeden Like-Nachbarn werden dann die entsprechenden Gewichte der Bewertungen summiert. Für
das obige Beispiel wären das folgende Summen:

\begin{table}[H]
	\begin{center}
		\begin{tabular}{llc}
			\toprule
			Like-Nachbar & Berechnung & Summe $s$ \\
			\midrule
			User1 & Rating 4 (1) & 1 \\
			User2 & Rating 5 (2) + Rating 3 (0.5) & 2.5 \\
			User3 & Rating 5 (2) & 2 \\
			\bottomrule
		\end{tabular}
	\end{center}
	\label{table:like-nachbar-bewertungssummen}
	\caption{Beispiel: Like-Nachbar-Bewertungssummen}
\end{table}

Die Menge aller Summen wird nachfolgend mit $S$ bezeichnet. Eine einzelne Summe wird mit $s$
bezeichnet.

$$S = \{ \textrm{s $|$ s ist die Bewertungs-Gewichtungssumme eines Like-Nachbars} \}$$

Schlussendlich werden diese Bewertungssummen verwendet, um die Gewichtung $g_{user}$, also die
Wichtigkeit, eines Like-Nachbars zu bestimmen. Da diese Gewichtung einen Wert zwischen 1 und 3 annehmen
soll, werden die Summen in diesem Bereich linear verteilt.

$$g_{user} = 1 + 2 \cdot \frac{s}{\max(S)}$$

Nachfolgend die Berechnung der Gewichtungen am Beispiel der Abbildung~\ref{fig:beispiel-like-nachbarn}
auf Seite~\pageref{fig:beispiel-like-nachbarn}.

\begin{table}[H]
	\begin{center}
		\begin{tabular}{lllc}
			\toprule
			Like-Nachbar & Summe $s$ & Berechnung & Gewichtung $g_{user}$ \\
			\midrule
			User1 & 1   & $1 + 2 \cdot \frac{1}{2.5}$   & \textbf{1.8} \\
			User2 & 2.5 & $1 + 2 \cdot \frac{2.5}{2.5}$ & \textbf{3}   \\
			User3 & 2   & $1 + 2 \cdot \frac{2}{2.5}$   & \textbf{2.6} \\
			\bottomrule
		\end{tabular}
	\end{center}
	\label{like-nachbar-multiplikatoren}
	\caption{Like-Nachbar-Multiplikatoren}
\end{table}


\newpage
\subsection{Schritt 2: Biertyp-Gewichtung}

Als nächster Schritt werden die Biertypen der beliebte Biere des Zielusers gewichtet. Folgende 
Illustration zeigt ein Beispiel für die Biertyp-Gewichtungs-Berechnung.

\begin{figure}[H]
	\begin{center}
		\begin{tikzpicture}[->,>=stealth',shorten >=1pt,auto,node distance=3cm,thick,scale=2.0,
				user/.style={circle,draw,font=\sffamily\large\bfseries},
				lager/.style={circle,draw,font=\sffamily\large\bfseries,color=orange,fill=orange!10!yellow!10},
				weizen/.style={circle,draw,font=\sffamily\large\bfseries,color=red!80,fill=red!10!orange!10},
				stout/.style={circle,draw,font=\sffamily\large\bfseries,color=green!60!black!80,fill=green!10},
			]

			\node[user] (Z) {Zieluser};

			\node[lager] (B1) [above of=Z] {Bier1};
			\node[lager] (B2) [above right of=Z] {Bier2};
			\node[weizen] (B3) [right of=Z] {Bier3};
			\node[weizen] (B4) [below right of=Z] {Bier4};
			\node[weizen] (B5) [below of=Z] {Bier5};
			\node[stout] (B6) [below left of=Z] {Bier6};
			\node[stout] (B7) [left of=Z] {Bier7};
			\node[stout] (B8) [above left of=Z] {Bier8};

			\path[every node/.style={font=\sffamily\small}]
				(Z)
					edge node {rates 4} (B1)
					edge node {rates 3} (B2)

					edge node {rates 3} (B3)
					edge node {rates 5} (B4)
					edge node {rates 4} (B5)

					edge node {rates 5} (B6)
					edge node {rates 4} (B7)
					edge node {rates 5} (B8)
				;

		\end{tikzpicture}
	\end{center}
	\caption{Beispiel Biertyp-Gewichtungen}
	\label{fig:beispiel-biertyp-gewichtungen}
\end{figure}

Die orangen Biere stehen für Lagerbier, die roten für Weizenbier und die grünen für Stout.

Zuerst nimmt man alle beliebten Biere des Zielusers.

$$B = \{ \textrm{b $|$ b ist ein beliebtes Bier des Zielusers} \}$$

Dann werden die Bewertungs-Gewichtungen gemäss Tabelle~\ref{table:bewertungs-gewichtungen}
(Seite~\pageref{table:bewertungs-gewichtungen}) auf die einzelnen Biere angewendet. Die Summe der
Bewertungs-Gewichtungen pro Biertyp ergeben die Biertyp-Bewertungssumme $s$.

$$s = \sum_{i=1}^k \textrm{gewichtung}(B_i)$$

Im obigem Beispiel (Abbildung~\ref{fig:beispiel-biertyp-gewichtungen}) wären das folgende Werte:

\begin{table}[H]
	\begin{center}
		\begin{tabular}{llc}
			\toprule
			Biertyp & Berechnung & Summe $s$ \\
			\midrule
			Stout  & $2 + 1 + 2$   & 5 \\
			Weizen & $0.5 + 2 + 1$ & 3.5 \\
			Lager  & $1 + 0.5$     & 1.5 \\
			\bottomrule
		\end{tabular}
	\end{center}
	\caption{Biertyp-Bewertungssummen}
	\label{table:biertyp-bewertungs-summen}
\end{table}

Von diesen Typen-Summen $T$ werden die fünf höchstbewerteten linear auf einer Skala zwischen 1 und
2 abgebildet. Dies ergibt die Biertyp-Gewichtung $g_{biertyp}$.

$$T = \{ \textrm{s $|$ s ist eine Biertyp-Bewertungssumme} \}$$
$$g_{biertyp} = 1 + \frac{s}{\max(T)}$$

Nachfolgend die konkreten Gewichtungen gemäss Abbildung~\ref{fig:beispiel-biertyp-gewichtungen}.

\begin{table}[H]
	\begin{center}
		\begin{tabular}{lllc}
			\toprule
			Biertyp & Summe $s$ & Berechnung & Gewichtung \\
			\midrule
			Stout  & 5   & $1 + \frac{5}{5}$   & \textbf{2} \\
			Weizen & 3.5 & $1 + \frac{3.5}{5}$ & \textbf{1.7}   \\
			Lager  & 1.5 & $1 + \frac{1.5}{5}$ & \textbf{1.3} \\
			\bottomrule
		\end{tabular}
	\end{center}
	\label{biertyp-gewichtungen}
	\caption{Biertyp-Gewichtungen}
\end{table}


\subsection{Schritt 3: Biergewichtung}

Nun werden alle beliebten Biere der Like-Nachbarn, abzüglich den Bieren die dem Zieluser schon bekannt sind,
gewichtet, um die Gesamtgewichtung $G_{bier}$ zu erhalten. Darin fliessen folgende drei Faktoren
ein:

\begin{itemize}
	\item Gewichtung der Like-Nachbarn $g_{user}$ (1 - 3)
	\item Gewichtung der Biertypen $g_{biertyp}$ (1 - 2)
	\item Gewichtung der Bier-Bewertungen durch Like-Nachbarn $g_{bier}$ (0.5 - 2)
\end{itemize}

Wenn ein Bier von mehreren Like-Nachbarn bewertet wurde, wird jeweils der Durchschnitt der
Like-Nachbar-Gewichtungen, sowie der Bier-Bewertungs-Gewichtungen verwendet.

Die Gesamtgewichtung wird folgendermassen berechnet:

$$G_{bier} = g_{user} \cdot g_{biertyp} \cdot g_{bier}$$

Damit ist der Empfehlungs-Algorithmus abgeschlossen.


\end{document}
