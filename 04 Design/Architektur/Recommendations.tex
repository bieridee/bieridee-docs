\documentclass[10pt,a4paper]{scrartcl}
\pagestyle{empty}
\usepackage{a4} % alternativ \usepackage{a4wide}
\usepackage[ngerman]{babel} % Neudeutsche Silbentrennung (mehrsprachiges Dokument)
\usepackage{parskip} % Skip indentation of first row
\usepackage{graphicx} % Graphics support
\usepackage{longtable} % Tables across several pages
\usepackage{booktabs}
\usepackage{hyperref} % Hyperlinks
\usepackage[automark]{scrpage2} %kopf/fusszeile
\usepackage[utf8x]{inputenc} % Unicode-Encoding

% amsmath (large fractions)
\usepackage{amsmath}

% Tikz
\usepackage{tikz}
\usetikzlibrary{arrows}
 
\linespread{1.3}

\author{Danilo Bargen, Christian Fässler, Jonas Furrer} 
\title{Bier-Empfehlungen\\ Projekt BierIdee}

\pagestyle{scrheadings}
\ihead{SE2 Projekte} %linke Kopfzeile
\ohead{BierIdee} %rechte Kopfzeile

\begin{document}

\begin{titlepage}
	\maketitle
	\vspace{120mm}
	\thispagestyle{empty} % Don't start page numbers on this page
\end{titlepage}

\tableofcontents
\newpage

\section*{Änderungshistorie}
\begin{tabular}{p{0.1\textwidth}p{0.15\textwidth}p{0.55\textwidth}p{0.1\textwidth}}
\toprule
\textbf{Version} & \textbf{Datum} & \textbf{Änderung} & \textbf{Person} \\  
\midrule
v1.0 & 15.05.2012 & Dokument erstellt & dbargen \\  
\bottomrule
\end{tabular} 
\newpage


\section{Einleitung}

Der Zweck dieses Dokumentes ist das Erläutern des verwendeten Bierempfehlungs-Algorithmus.


\section{Glossar}

In diesem Dokument werden folgende Bezeichnungen verwendet:

\begin{description}
	\item[Zielbenutzer] Der Benutzer, der sich eine Empfehlung ausstellen lassen will
	\item[Like-Nachbarn] Andere User, deren Biergeschmack sich mit dem des Zielbenutzers überschneidet
\end{description}


\section{Verfahren}

Um die Empfehlungen zu berechnen, werden verschiedene Faktoren berechnet und unterschiedlich
gewichtet.

\subsection{Schritt 1: Like-Nachbarn}

Um Like-Nachbarn zu berechnen, geht man von allen Bieren aus, die vom Zielbenutzer eine Bewertung
von 3 oder höher erhalten haben.

\vspace{10mm}

\begin{center}
	\begin{tikzpicture}[->,>=stealth',shorten >=1pt,auto,node distance=3cm,thick,scale=2.0,
			zieluser/.style={circle,draw,font=\sffamily\large\bfseries},
			beer/.style={circle,draw,font=\sffamily\large\bfseries},
			user/.style={circle,draw,font=\sffamily\large\bfseries,color=red,fill=red!10!orange!10},
		]

		\node[zieluser] (Z) at (3,3) {Zieluser};

		\node[beer] (B1) at (2,2) {Bier1};
		\node[beer] (B2) at (4,2) {Bier2};

		\node[user] (U1) at (1,1) {User1};
		\node[user] (U2) at (3,1) {User2};
		\node[user] (U3) at (5,1) {User3};

		\path[every node/.style={font=\sffamily\small}]
			(Z)
				edge node [swap] {rates 4} (B1)
				edge node {rates 5} (B2)
			(U1)
				edge node {rates 4} (B1)
			(U2)
				edge node {rates 5} (B1)
				edge node [swap] {rates 3} (B2)
			(U3)
				edge node [swap] {rates 5} (B2);

	\end{tikzpicture}
\end{center}

\newpage

Die Bewertungen werden dabei abgestuft gewichtet:

\begin{tabular}{cc}
	\toprule
	Bewertung & Multiplikator \\
	\midrule
	3 & 0.5 \\
	4 & 1 \\
	5 & 2 \\
	\bottomrule
\end{tabular}

\vspace{5mm}

Für jeden Like-Nachbarn werden dann die entsprechenden Gewichte der Bewertungen summiert. Für
das obige Beispiel wären das folgende Summen:

\begin{tabular}{llc}
	\toprule
	Like-Nachbar & Berechnung & Summe \\
	\midrule
	User1 & Rating 4 (1) & 1 \\
	User2 & Rating 5 (2) + Rating 3 (0.5) & 2.5 \\
	User3 & Rating 5 (2) & 2 \\
	\bottomrule
\end{tabular}

\vspace{5mm}

Schlussendlich werden diese Bewertungssummen verwendet, um die Gewichtung eines Like-Nachbars zu bestimmen.
Da diese Gewichtung einen Wert zwischen 1-3 annehmen soll, werden die Summen linear daziwschen verteilt.

$$usergewichtung = 1 + 2 \cdot \frac{s}{m}$$

Dabei ist $s$ die oben berechnete Bewertungssumme und $m$ ist die höchste vorkommende Summe, also max($summen$).

\begin{tabular}{lllc}
	\toprule
	Like-Nachbar & Summe & Berechnung & Gewichtung \\
	\midrule
	User1 & 1   & $1 + 2 \cdot \frac{1}{2.5}$   & 1.8 \\
	User2 & 2.5 & $1 + 2 \cdot \frac{2.5}{2.5}$ & 3   \\
	User3 & 2   & $1 + 2 \cdot \frac{2}{2.5}$   & 2.6 \\
	\bottomrule
\end{tabular}


\subsection{Schritt 2: Biergewichtung}

Als nächster Schritt werden die Lieblingsbiere aller Like-Nachbarn aus der Datenbank geholt. Als
Lieblingsbiere gelten wiederum alle Biere mit einer Bewertung von 3-5.

Von allen diesen Bieren werden die Bewertungen erneut gemäss der Multiplikator-Tabelle aus Schritt
1 gewichtet. Zusätzlich wird die Usergewichtung als Multiplikator verwendet.

$$rg = rm \cdot ug$$

Dabei ist $rg$ die Rating-Gewichtung, $rm$ der Rating-Multiplikator und $ug$ die Usergewichtung.

Um die Biergewichtung zu errechnen, muss man nun einfach die Gewichtungen der einzelnen Ratings
addieren.

$$bg = rg_1 + rg_2 + \cdots + rg_n$$


\end{document}
