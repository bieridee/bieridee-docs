\documentclass[10pt,a4paper]{scrartcl}
\pagestyle{empty}
\usepackage{a4} % alternativ \usepackage{a4wide}
\usepackage[ngerman]{babel} % Neudeutsche Silbentrennung (mehrsprachiges Dokument)
\usepackage{parskip} % Skip indentation of first row
\usepackage{graphicx} % Graphics support
\usepackage{longtable} % Tables across several pages
\usepackage{booktabs}
\usepackage{hyperref} % Hyperlinks
\usepackage[automark]{scrpage2} %kopf/fusszeile
\usepackage[utf8x]{inputenc} % Unicode-Encoding
 
\linespread{1.3}

\author{Danilo Bargen, Christian Fässler, Jonas Furrer} 
\title{Authentifizierung\\ Projekt BierIdee}

\pagestyle{scrheadings}
\ihead{SE2 Projekte} %linke Kopfzeile
\ohead{BierIdee} %rechte Kopfzeile

\begin{document}

\begin{titlepage}
	\maketitle
	\vspace{120mm}
	\thispagestyle{empty} % Don't start page numbers on this page
\end{titlepage}

\tableofcontents
\newpage

\section*{Änderungshistorie}
\begin{tabular}{p{0.1\textwidth}p{0.15\textwidth}p{0.55\textwidth}p{0.1\textwidth}}
\toprule
\textbf{Version} & \textbf{Datum} & \textbf{Änderung} & \textbf{Person} \\  
\midrule
v1.0 & 14.04.2012 & Dokument erstellt & cfaessle \\  
\hline 
v1.1 & 16.04.2012 & Review & dbargen \\
\bottomrule
\end{tabular} 
\newpage


\section{Einleitung}
Der Zweck dieses Dokumentes ist das Erläutern der verwendeten Authentifizierungsmechanismen für die
REST API im Projekt Bieridee. Eine wichtige Voraussetzung ist, dass auch dieser Aspekt
\textit{RESTful} implementiert ist, wobei vor allem das Prinzip der \textit{Statelessness} zum
Tragen kommt.


\section{Verfahren}
Zur Authentifizierung kommt das HMAC\footnote{Hash-based Message Authentication Code, siehe
http://en.wikipedia.org/wiki/HMAC} Verfahren gemäss
RFC 2104\footnote{http://www.ietf.org/rfc/rfc2104.txt} zum Einsatz. Alle Requests (Messages) werden
mittels Passwort und einer Hashfunktion signiert.

\subsection{Registrierung}
Der Benutzer registriert einen neuen Benutzeraccount.
Zum Benutzeraccount gehört ein Passwort, welches dem Server hashed (SHA-256) übermittelt wird.
Der Hashwert wird nachfolgend "`Secret"' genannt. 

\subsection{HTTP Request vorbereiten}
Der Client bereitet seinen HTTP Request vor. 

\subsection{Signatur berechnen}
Über ein definiertes Set der HTTP Request Attribute wird eine Signatur (Hashwert) gebildet. Für
die Bildung dieses Hashwertes wird gemäss RFC 2104 das Secret verwendet.

Für das genaue Berechnungsverfahren des HMAC, siehe Anhang \textit{\ref{hmac-calc}}.

\subsection{Authorization Header setzen}
Der Request wird mit einem zusätzlichen HTTP Header versehen, welcher den Benutzernamen sowie die
generierte Signatur beinhaltet:

\begin{verbatim}
Authorization: <username>:<signatur>
\end{verbatim}

Anschliessend wird der Request abgesendet.

\subsection{Signatur verifizieren}
Der Server empfängt den Request und extrahiert den \texttt{Authorization} Header. Anhand des
Benutzernamens kann er in der Datenbank das gemeinsame Secret ausfindig machen und über das selbe
Set von Attributen auch die Signatur berechnen.

\subsection{Benutzer authentifizieren}
Die beiden Signaturen werden verglichen. Stimmen Sie überein, ist der User erfolgreich
authentifiziert.


\section{Technische Details}

\subsection{Hashing Algorithmus}
Als Hash Algorithmus wird
SHA-256\footnote{http://csrc.nist.gov/publications/fips/fips180-3/fips180-3\_final.pdf} verwendet.

\subsection{Speicherung der Passwörter}
Die Passwörter werden -- sowohl in der Datenbank wie auch im Client -- ebenfalls hashed
gespeichert. Somit haben alle Passwörter ein einheitliches Format. Keine Sonderzeichen, gleiche
Länge. Für die HMAC Bildung werden die Hashwerte dieser Passwörter verwendet.


\section{Sicherheit}

\subsection{Aushandlung gemeinsames Secret}
Der einzige Schwachpunkt der angedachten Methode ist der initiale Austausch des gemeinsamen Secrets.

\subsection{Hashing Algorithmus}
Der SHA-256 Algorithmus ist ein standardisierter und weit verbreiteter Hashing-Algorithmus. Er
bietet mehr Sicherheit als MD5 und SHA-1 und wird desweiteren von Java out-of-the-box unterstützt.


\newpage\section{Anhang}

\subsection{\label{hmac-calc}Berechnung des HMAC}

Der HMAC wird aus der Nachricht $N$ und einem geheimen Schlüssel $K$ mittels der Hash-Funktion
$\mathrm{H}$ nach RFC 2104 wie folgt berechnet. $K$ wird auf die Blocklänge $B$ der Hash-Funktion (512
Bit für die meisten gängigen Hash-Funktionen) aufgefüllt. Falls die Länge von $K$ grösser als die
Blocklänge der Hash-Funktion ist, wird $K$ durch $\mathrm{H}(K)$ ersetzt.
$$\mathrm{HMAC}_K(N) = \mathrm{H}\left(\left(K \oplus opad \right) \;||\; \mathrm{H}\left(\left(K
\oplus ipad \right) \;||\; N\right)\right)$$

Die Werte $opad$ und $ipad$ sind dabei Konstanten, $\oplus$ steht
für die bitweise XOR-Operation und $\;||\;$ für die Verknüpfung durch einfaches
Zusammensetzen (Konkatenation).

Nach RFC 2104 sind beide Konstanten wie folgt definiert:
$$opad = \underbrace{\mathrm{0x5C}\dots \mathrm{0x5C}}_{B\textrm{-mal}} \textrm{ und } ipad =
\underbrace{\mathrm{0x36}\dots \mathrm{0x36}}_{B\textrm{-mal}}$$

\end{document}
