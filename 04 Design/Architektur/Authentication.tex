\documentclass[10pt,a4paper]{scrartcl}
\pagestyle{empty}
\usepackage{a4} % alternativ \usepackage{a4wide}
\usepackage[ngerman]{babel} % Neudeutsche Silbentrennung (mehrsprachiges Dokument)
\usepackage{parskip} % Skip indentation of first row
\usepackage{graphicx} % Graphics support
\usepackage{longtable} % Tables across several pages
\usepackage{booktabs}
\usepackage{hyperref} % Hyperlinks
\usepackage[automark]{scrpage2} %kopf/fusszeile
\usepackage[utf8x]{inputenc} % Unicode-Encoding
 
\linespread{1.3}

\author{Danilo Bargen, Christian Fässler, Jonas Furrer} 
\title{Authentifizierung\\ Projekt BierIdee}

\pagestyle{scrheadings}
\ihead{SE2 Projekte} %linke Kopfzeile
\ohead{BierIdee} %rechte Kopfzeile

\begin{document}

\begin{titlepage}
	\maketitle
	\vspace{120mm}
	\thispagestyle{empty} % Don't start page numbers on this page
\end{titlepage}

\tableofcontents
\newpage

\section*{Änderungshistorie}
\begin{tabular}{p{0.1\textwidth}p{0.15\textwidth}p{0.55\textwidth}p{0.1\textwidth}}
\toprule
\textbf{Version} & \textbf{Datum} & \textbf{Änderung} & \textbf{Person} \\  
\midrule
v1.0 & 14.04.2012 & Dokument erstellt & cfa \\  
\hline 
v1.1 & 00.00.2012 & -- & kürzel \\
\hline 
v1.2 & 00.00.2012 & -- & kürzel \\
\bottomrule
\end{tabular} 
\newpage

\section{Einleitung}
Der Zweck dieses Dokumentes ist das aufzeigen der verwendeten Authentifizierungsmechanismen für die REST API im Projekt Bieridee. Eine wichtige Voraussetzung ist, dass auch dieser Aspekt Restful implementiert ist, sprich "Zustandslos".
\section{Verfahren}
Zur Authentifizierung kommt das Hash-MAC Verfahren zum Einsatz. Alle Requests (Messages) werden mittels Passwort signiert.
\subsubsection{Registrierung}
Der Benutzer registriert einen neuen Benutzeraccount.
Der Benutzeraccount besitzt ein Passwort, welches dem Server gehasht (SHA-1) übermittelt wird.
Der Hashwert wird nachfolgend "Secret" genannt. 
\subsubsection{Schritt 1}
Der Client bereitet seinen REST Request vor. 
\subsubsection{Schritt 2}
Über ein definiertes Set von Attributen wird ein Hashwert (Signatur) gebildet. Für die Bildung dieses Hashwertes wird gemäss RFC2104 das Secret verwendet.
\subsubsection{Schritt 3}
Der Request wird mit einem zusätzlichen Header Versehen der den Benutzernamen sowie die generierte Signatur beinhaltet. Authorization: <username>:<signatur>
Anschliessend wird der Request abgesendet.
\subsubsection*{Schritt 4}
Der Server empfängt den Request und extrahiert den Authorization Header. Anhand der des Benutzernamens kann er in der Datenbank das gemeinsame Secret ausfindig machen und über das selbe Set von Attributen auch die Signatur berechnen.
\subsection*{Schritt 5}
Die beiden Signaturen werden verglichen, stimmen Sie überein ist der User erfolgreich authentifiziert.

\section{Technische Details}
\subsection*{Hashing Algorithmus}
Als Hash Algorithmus wird SHA-1 verwendet.
\subsection*{Speicherung der Passwörter}
Die Passwörter werden ebenfalls gehast abgespeichert. Somit haben alle Passwörter ein einheitliches Format. Keine Sonderzeichen, gleiche Länge. Für die HMAC Bildung werden die Hashwerte dieser Passwörter verwendet. Das Eigentliche Passwort ist also nicht das ursprünglich eingegebene, sondern der Hashwert davon.
\section*{Sicherheit}
\subsection*{Aushandlung gemeinsames Secret}
Der Einzige Schwachpunkt der angedachten Methode ist der initiale Austausch vom gemeinsamen Secret.
\subsection*{Hash Algorithmus}
Der SHA-1 Algorithmus ist ein Standartisierter und weitverbreiteter Hash Algorithmus.
\end{document}
