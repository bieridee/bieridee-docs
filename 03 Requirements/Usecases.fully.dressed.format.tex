\documentclass[10pt,a4paper]{scrartcl}
\pagestyle{empty}
\usepackage{a4} % alternativ \usepackage{a4wide}
\usepackage[ngerman]{babel} % Neudeutsche Silbentrennung (mehrsprachiges Dokument)
\usepackage{parskip} % Skip indentation of first row
\usepackage{graphicx} % Graphics support
\usepackage{longtable} % Tables across several pages
\usepackage{booktabs}
\usepackage{hyperref} % Hyperlinks
\usepackage[automark]{scrpage2} %kopf/fusszeile
\usepackage[utf8x]{inputenc} % Unicode-Encoding
 
\linespread{1.3}

\author{Danilo Bargen, Christian Fässler, Jonas Furrer} 
\title{Usecases Projekt BierIdee}
\subtitle{Fully Dressed format}

\pagestyle{scrheadings}
\ihead{SE2 Projekte} %linke Kopfzeile
\ohead{BierIdee} %rechte Kopfzeile

\begin{document}

\begin{titlepage}
	\maketitle
	\vspace{120mm}
	\thispagestyle{empty} % Don't start page numbers on this page
\end{titlepage}

\section{Änderungshistorie}
\begin{tabular}{p{0.1\textwidth}p{0.15\textwidth}p{0.55\textwidth}p{0.1\textwidth}}
\toprule
\textbf{Version} & \textbf{Datum} & \textbf{Änderung} & \textbf{Person} \\  
\midrule
v1 & 21.03.2012 & Dokument erstellt & cfaessle \\  
\hline 
v1.1 & 01.04.2012 & Korrekturen nach Review & jfurrer \\
\bottomrule
\end{tabular} 
\newpage

\section*{UC01: User registriert sich }

Ein User registriert sich um einen Benutzerzugang (Benutzername, Passwort) zu erhalten. Der Benutzer erhält eine Bestätigungs-Email für die Registration. Er wird nach der Registration automatisch eingeloggt und kann bereits seinen Bierkonsum tracken. Erst wenn er die Email-Adresse per Link bestätigt hat, kann er auch neue Biere erfassen und ändern.


\begin{description}
\item[Scope] BierIdee System
\item[Level] User Goal
\item[Primary Actor] Der User, der sich für die Benutzung der App registrieren will.
\item[Features] F02 (\textit{siehe `Anforderungsspezifikation'})
\end{description}


\subsection*{Stakeholders and Interests}

\begin{description}
\item[User] möchte einen Benutzeraccount erstellen um sich am System anmelden zu können.
\item[System] hat das Ziel dem Benutzer eine intuitives Interface zur Registrierung zur Verfügung zu stellen. Es bestätigt dem User eine erfolgreiche Registrierung.
\end{description}


\subsection*{Preconditions}

\begin{itemize}
\item Die Android App muss auf dem Benutzergerät installiert sein.
\item Das Benutzergerät hat eine bestehende Internetverbindung.
\end{itemize}


\subsection*{Success Guarantee / Postconditions}


\subsection*{Main Success Scenario}

\begin{enumerate}
\item Der Benutzer startet die Android App.
\item Er wählt die Option zur Neuregistrierung / Eröffnung eines Benutzerkontos
\item Der Benutzer wird aufgefordert einen Benutzernamen, Passwort und Emailadresse einzugeben.
\item Das System meldet eine erfolgreiche Registrierung und gibt einen Hinweis auf die Aktivierungs-Email.
\item Der Benutzer erhält die Email mit einem Aktivierungslink
\item Mittels Aktivierungslink aktiviert der Benutzer sein Konto für die Verwendung. 
\end{enumerate}


\subsection*{Extensions}

\begin{description}
\item[3a] Der gewünschte Benutzername oder Emailadresse ist bereits vergeben.
	\begin{enumerate}
	\item Der Benutzer wird aufgefordert eine andere Emailadresse und/oder Benutzernamen 		einzugeben.
	\end{enumerate}

\end{description}


\subsection*{Special Requirements}

\begin{itemize}
\item Das Registrieren eines Benutzerkontos sollte direkt nach dem Start der Applikation möglich sein. 
\end{itemize}



\subsection*{Frequency of Occurrence}

Die Neuregistrierung ist die Grundlage für alle weiteren Interaktionen mit dem System. Daher wird Sie von jedem Benutzer durchgeführt, allerdings nur einmal in der ganzen Zeit der Nutzung. Die Registrationen werden sich zu Anfang häufen und dann in der Zahl immer geringer werden.


\subsection*{Open Issues}

\begin{itemize}
\item Möglicherwiese wird der Login über Drittsysteme wie Facebook oder Twitter möglich sein.
\end{itemize}


\section*{UC05: User bewertet Bier}
Der Benutzer kann ein Bier auswählen und es mit einer persönlichen Bewertung versehen. Wenn der Benutzer bereits eine Bewertung erfasst hat, wird diese angezeigt und kann angepasst werden.

\begin{description}
\item[Scope] BierIdee System
\item[Level] User Goal
\item[Primary Actor] Der Benutzer welcher ein Bier bewerten will
\item[Features] F05 (\textit{siehe `Anforderungsspezifikation'})
\end{description}


\subsection*{Stakeholders and Interests}

\begin{description}
\item[User] möchte ein Bier gemäss seinen Vorstellungen bewerten.
\item[System] hat das Ziel dem Benutzer eine intuitives Interface zur Verfügung zu stellen und die Bewertung des Benutzers abzuspeichern sowie anzuzeigen.
\end{description}


\subsection*{Preconditions}

\begin{itemize}
\item Der Benutzer ist am System angemeldet (Login).
\item Im System sind bewertbare Biere erfasst.
\end{itemize}


\subsection*{Success Guarantee / Postconditions}
\begin{itemize}
\item Die Bewertung des Benutzers ist im System abgespeichert
\end{itemize}


\subsection*{Main Success Scenario}

\begin{enumerate}
\item Der Benutzer wählt ein Bier aus einer Liste von Bieren ein Bier aus.
\item Er kann anhand einer Skala das Bier bewerten.
\item Das System meldet dem Benutzer die erfolgreiche Bewertung
\end{enumerate}


\subsection*{Extensions}

\begin{description}
\item[1a] Die Auswahlliste ist sehr lang.
	\begin{enumerate}
	\item Der Benutzer kann die Auswahlliste gezielt filtern.
	\end{enumerate}
\item[2a] Der Benutzer hat das Bier bereits bewertet.
	\begin{enumerate}
	\item Der Benutzer sieht seine bereits erfasste Bewertung
	\item Er kann seine Bewertung ändern.
	\end{enumerate}

\end{description}


\subsection*{Special Requirements}

\begin{itemize}
\item Das Suchen in der Auswahlliste soll effizient sein. Denkbar wäre eine Alphabetische Sortierung mit Buchstaben als Shortcuts oder sogar ein Filterfeld mit Autocompletion.
\end{itemize}


\subsection*{Frequency of Occurrence}

Das Bewerten von Bieren ist eine der zentralen Aktionen die Benutzer ausführen sollen.
Auf Basis dieser Bewertungen werden individuelle Empfehlungen generiert. Die Nutzung wird sich zu Abendzeiten und an Wochenenden häufen.

\subsection*{Open Issues}

\begin{itemize}
\item Wie werden Biere zur Auswahl angeboten. Mittels scrollbarer Liste oder mittels Filterfeld oder einer Kombination von beidem.
\end{itemize}



\section*{UC07: User ruft Bier-Empfehlungen ab}
Der Benutzer kann auf sein Profil bezogene Empfehlungen für ihm bisher unbekannte (unbewertete) Biersorten abrufen.

\begin{description}
\item[Scope] BierIdee System
\item[Level] User Goal
\item[Primary Actor] Der Benutzer welcher individuelle Empfehlungen erhalten möchte.
\item[Features] F06 (\textit{siehe `Anforderungsspezifikation'})
\end{description}


\subsection*{Stakeholders and Interests}

\begin{description}
\item[User] möchte eine Bierempfehlung anhand seines Profiles erhalten.
\item[System] hat das Ziel dem Benutzer übersichtliche und auf sein Profil abgestimmte  Bierempfehlungen zu geben (das Profil basiert auf seinen erfassten Aktivitäten und Bewertungen).
\end{description}


\subsection*{Preconditions}

\begin{itemize}
\item Der Benutzer ist am System angemeldet.
\end{itemize}


\subsection*{Success Guarantee / Postconditions}
\begin{itemize}
\item Der Benutzer erhält Empfehlungen basierend auf seinen erfassten Bewertungen und Aktivitäten.
\end{itemize}


\subsection*{Main Success Scenario}

\begin{enumerate}
\item Der Benutzer wählt die Option `Empfehlungen anzeigen'.
\item Das System listet dem Benutzer die generierten Empfehlungen auf.
\item Der Benutzer kann diese Biere bewerten (siehe UC05).
\end{enumerate}


\subsection*{Extensions}

\begin{description}
\item[1a] Der Benutzer hat noch keine oder zuwenige Bewertungen oder Aktivitäten erfasst.
	\begin{enumerate}
	\item Dem Benutzer werden keine indivuellen Empfehlungen angezeigt, sondern nur die beliebtesten Biere.
	\item Das System meldet dem Benutzer, dass für individuelle Empfehlungen mehr Aktivität benötigt wird.
	\end{enumerate}

\end{description}


\subsection*{Special Requirements}

\begin{itemize}
\item Der Wechsel in den Use Case UC05 soll einfach möglich sein. D.h schnell, ohne grossen Kontextwechsel/Bedienungsaufwand.
\end{itemize}



\subsection*{Frequency of Occurrence}

Das Einsehen von Empfehlungen ist eine Zentrale Funktionalität. Daher wird die Nutzung häufig auftreten. 

\subsection*{Open Issues}

\begin{itemize}
\item Keine.
\end{itemize}


\section*{UC10: User erfasst Konsum}
Der Benutzer kann eine Konsumaktivität erfassen.

\begin{description}
\item[Scope] BierIdee System
\item[Level] User Goal
\item[Primary Actor] Der Benutzer welcher einen Konsum erfassen möchte.
\item[Features] F10 (\textit{siehe `Anforderungsspezifikation'})
\end{description}


\subsection*{Stakeholders and Interests}

\begin{description}
\item[User] möchte einen Bierkonsum im System erfassen.
\item[System] hat das Ziel den Konsum abzuspeichern und mit dem Benutzer zu verknüpfen.
\end{description}


\subsection*{Preconditions}

\begin{itemize}
\item Der Benutzer ist am System angemeldet.
\item Es sind Biere im System erfasst.
\end{itemize}


\subsection*{Success Guarantee / Postconditions}
\begin{itemize}
\item Der Benutzer hat einen Bierkonsum erfasst. Der Konsum ist in der Timeline sichtbar und wirkt sich auf Empfehlungen für den Benutzer aus.
\end{itemize}


\subsection*{Main Success Scenario}

\begin{enumerate}
\item Der Benutzer wählt die Option Konsum erfassen.
\item Der Benutzer wählt ein Bier aus einer Liste aus.
\item Der Benutzer wählt optional den Ort des Konsums.
\end{enumerate}


\subsection*{Extensions}

\begin{description}
\item[1-2a] Der Benutzer sieht eine Empfehlung an.
	\begin{enumerate}
	\item Der Benutzer kann direkt auf die Empfehlung einen Konsum erfassen \textit{(siehe UC07)}.
	\end{enumerate}
\item[2b] Die Auswahlliste ist sehr lang.
	\begin{enumerate}
	\item Der Benutzer kann auch gezielt nach Biernamen suchen.
	\end{enumerate}
\item[4a] Der Benutzer kann vor dem Speichern des Konsums das Bier auch gleich bewerten \textit{(siehe UC05)}.
\end{description}


\subsection*{Special Requirements}

\begin{itemize}
\item Die Erfassung eines Konsum soll aus jeder dafür sinnvollen Ansicht der App möglich sein, z.B. Bier ansehen, Bier bewerten.
\end{itemize}



\subsection*{Frequency of Occurrence}

Das Erfassen von konsumierten Getränken ist eine Kernfunktionalität welche aber nicht verwendet werden muss um Empfehlungen zu erhalten. Die Nutzung wird sich zu Abendzeiten und an Wochenenden häufen.

\subsection*{Open Issues}

\begin{itemize}
\item Keine.
\end{itemize}

\end{document}