\documentclass[10pt,a4paper]{scrartcl}
\pagestyle{empty}
\usepackage{a4} % alternativ \usepackage{a4wide}
\usepackage[ngerman]{babel} % Neudeutsche Silbentrennung (mehrsprachiges Dokument)
\usepackage{parskip} % Skip indentation of first row
\usepackage{graphicx} % Graphics support
\usepackage{longtable} % Tables across several pages
\usepackage{hyperref} % Hyperlinks
\usepackage[automark]{scrpage2} %kopf/fusszeile
\usepackage[utf8x]{inputenc} % Unicode-Encoding
 
\linespread{1.3}

\author{Danilo Bargen, Christian Fässler, Jonas Furrer} 
\title{Usecases Projekt BierIdee}
\subtitle{Fully Dressed format}

\pagestyle{scrheadings}
\ihead{SE2 Projekte} %linke Kopfzeile
\ohead{BierIdee} %rechte Kopfzeile

\begin{document}

\begin{titlepage}
	\maketitle
	\vspace{120mm}
	\thispagestyle{empty} % Don't start page numbers on this page
\end{titlepage}



\section*{UC01: User registriert sich }

Ein User kann sich registrieren um einen Benutzerzugang (Benutzername,Passwort) zu erhalten. Der Benutzer erhält eine Bestätigungs-email der Registration.
Diese Registration muss er mittels Link bestätigen und kann sich danach einloggen.



\begin{description}
\item[Scope] BierIdee System
\item[Level] User Goal
\item[Primary Actor] Der User, der sich für die Benutzung der App registrieren will.
\end{description}


\subsection*{Stakeholders and Interests}

\begin{description}
\item[User] möchte einen Benutzeraccount erstellen um sich am System anmelden zu können.
\item[System] hat das Ziel dem Benutzer eine intuitives Interface zur Registrierung zur Verfügung zu stellen. Es bestätigt dem User eine erfolgreiche Registrierung.
\end{description}


\subsection*{Preconditions}

\begin{itemize}
\item Die Android App muss auf dem Benutzergerät installiert sein.
\end{itemize}


\subsection*{Success Guarantee / Postconditions}


\subsection*{Main Success Scenario}

\begin{enumerate}
\item Der Benutzer startet die Android App.
\item Er wählt die Option zur Neuregistrierung / Eröffnung eines Benutzerkontos
\item Der Benutzer wird aufgefordert einen Benutzernamen, Passwort und Emailadresse einzugeben.
\item Das System meldet eine erfolgreiche Registrierung.
\item Der Benutzer erhält eine Email mit einem Aktivierungslink
\item Mittels Aktivierungslink aktiviert der Benutzer sein Konto für die Verwendung. 
\end{enumerate}


\subsection*{Extensions}

\begin{description}
\item[3a] Der gewünschte Benutzername oder Emailadresse ist bereits vergeben.
	\begin{enumerate}
	\item Der Benutzer wird aufgefordert eine andere Emailadresse und/oder Benutzernamen 		einzugeben.
	\end{enumerate}
\item[4a] Der Benutzer erhält keine Email mit einem Aktivierungslink.
	\begin{enumerate}
	\item Die Registrierung kann erneut durchgeführt werden mit Benutzernamen, die noch nicht aktiviert wurden.
	\end{enumerate}

\end{description}


\subsection*{Special Requirements}

\begin{itemize}
\item Das Neuregistrieren des eines Benutzerkontos sollte direkt nach dem Start der Applikation möglich sein. 
\end{itemize}



\subsection*{Frequency of Occurrence}

Die Neuregistrierung ist die Grundlage für alle weiteren Interaktionen mit dem System. Daher wird Sie von jedem Benutzer durchgeführt, dies entspricht einer Benutzung von 100%.


\subsection*{Open Issues}

\begin{itemize}
\item Werden evtl. Benutzeraccounts aus anderen Systeme wie Facebook, Twitter erlaubt zum Login.
\end{itemize}


\section*{UC05: User bewertet Bier}
Der Benutzer kann ein Bier auswählen und jenes mit einer persönlichen Bewertung versehen. Wenn der Benutzer bereits eine Bewertung erfasst hat, wird diese angezeigt und kann angepasst werden.

\begin{description}
\item[Scope] BierIdee System
\item[Level] User Goal
\item[Primary Actor] Der Benutzer welcher ein Bier bewerten will
\end{description}


\subsection*{Stakeholders and Interests}

\begin{description}
\item[User] möchte ein Bier gemäss seinen Idealvorstellungen bewerten.
\item[System] hat das Ziel dem Benutzer eine intuitives Interface zur Verfügung zu stellen. Und die Bewertung des Benutzers abzuspeichern.
\end{description}


\subsection*{Preconditions}

\begin{itemize}
\item Die Android App muss auf dem Benutzergerät installiert sein.
\item Der Benutzer hat ein Benutzerkonto
\item Der Benutzer ist am System angemeldet (Login)
\item Im System sind bewertbare Biere erfasst
\item Das Benutzergerät verfügt über eine Internetverbindung
\end{itemize}


\subsection*{Success Guarantee / Postconditions}
\begin{itemize}
\item Die Bewertung des Benutzers ist im System abgespeichert
\end{itemize}


\subsection*{Main Success Scenario}

\begin{enumerate}
\item Wählt in einer Liste von Bieren ein Bier aus
\item Er kann anhand einer Skala das Bier bewerten. (Bspw. 1-5)
\item Im Hintergrund wird die Bewertung abgespeichert
\item Das System meldet dem Benutzer die erfolgreiche Bewertung
\end{enumerate}


\subsection*{Extensions}

\begin{description}
\item[1a] Die Auswahlliste ist sehr lang.
	\begin{enumerate}
	\item Der Benutzer kann auch gezielt nach Biernamen suchen.
	\end{enumerate}
\item[2a] Der Benutzer hat das Bier bereits bewertet.
	\begin{enumerate}
	\item Der Benutzer sieht seine bereits erfasste Bewertung
	\item Er kann seine Bewertung ändern.
	\end{enumerate}

\end{description}


\subsection*{Special Requirements}

\begin{itemize}
\item Das Suchen in der Auswahlliste soll effizient sein. Denkbar wäre eine Alphabetische Sortierung  mit Buchstaben Shortcuts oder sogar ein Filterfeld mit Autocompletion.
\end{itemize}



\subsection*{Frequency of Occurrence}

Das Bewerten von Bieren ist eine der zentralen Anwendungen die Benutzer ausführen sollen.
Auf Basis dieser Bewertungen werden individuelle Empfehlungen generiert. Hier ergibt sich eine Benutzung von 100%.

\subsection*{Open Issues}

\begin{itemize}
\item Wie werden Biere zur Auswahl angeboten. Mittels scrollbarer Liste oder mittels Filterfeld.
\end{itemize}



\section*{UC07: User ruft Bier-Empfehlungen ab}
Der Benutzer kann auf sein Profil bezogene Empfehlungen für ihm bisher unbekannte (unbewertet) Biersorten abrufen.

\begin{description}
\item[Scope] BierIdee System
\item[Level] User Goal
\item[Primary Actor] Der Benutzer welcher individuelle Empfehlungen erhalten möchte.
\end{description}


\subsection*{Stakeholders and Interests}

\begin{description}
\item[User] möchte ein Bier gemäss seines Profiles empfholen erhalten.
\item[System] hat das Ziel dem Benutzer übersichtlich auf sein Profil abgestimmte bdas Profil basiert auf seinen erfassten Aktivitäten und Bewertungen) Bier Empfehlungen zu erhalten.
\end{description}


\subsection*{Preconditions}

\begin{itemize}
\item Der Benutzer besitzt ein Benutzerkonto
\item Der Benutzer ist am System angemeldet.
\item Das Benutzergerät hat eine bestehende Internetverbindung.
\end{itemize}


\subsection*{Success Guarantee / Postconditions}
\begin{itemize}
\item Der Benutzer erhält Empfehlungen basieren auf seinen erfassten Bewertungen und Aktivitäten.
\end{itemize}


\subsection*{Main Success Scenario}

\begin{enumerate}
\item Der Benutzer wählt die Option Empfehlungen anzeigen
\item Das System listet dem Benutzer die generierten Empfehlungen auf.
\item Der Benutzer kann diese Biere bewerten (siehe UC05)
\end{enumerate}


\subsection*{Extensions}

\begin{description}
\item[1a] Der Benutzer hat noch keine Bewertungen oder Aktivitäten erfasst.
	\begin{enumerate}
	\item Dem Benutzer werden keine Empfehlungen angezeigt. Das System meldet dem Benutzer, dass es keine Empfehlungen machen kann
	\end{enumerate}

\end{description}


\subsection*{Special Requirements}

\begin{itemize}
\item Der Wechsel in den Use Case UC05 soll einfach möglich sein. D.h schnell, ohne grossen Kontextwechsel/Bedienungsaufwand.
\end{itemize}



\subsection*{Frequency of Occurrence}

Das Einsehen von Empfehlungen ist eine Zentrale Funktionalität. Daher auch eine Benutzung von 100%.

\subsection*{Open Issues}

\begin{itemize}
\item nichts
\end{itemize}


\section*{UC10: User erfasst oder trackt Konsum}
Der Benutzer kann eine Konsum Aktivität erfassen.

\begin{description}
\item[Scope] BierIdee System
\item[Level] User Goal
\item[Primary Actor] Der Benutzer welcher einen Konsum erfassen möchte.
\end{description}


\subsection*{Stakeholders and Interests}

\begin{description}
\item[User] möchte ein Konsumiertes Getränkt im System erfassen.
\item[System] hat das Ziel den Konsum abzuspeichern und mit dem Benutzer zu verknüpfen.
\end{description}


\subsection*{Preconditions}

\begin{itemize}
\item Der Benutzer besitzt ein Benutzerkonto
\item Der Benutzer ist am System angemeldet.
\item Es sind Biere im System erfasst.
\item Das Benutzergerät hat eine bestehende Internetverbindung.
\end{itemize}


\subsection*{Success Guarantee / Postconditions}
\begin{itemize}
\item Der Benutzer hat ein Konsumiertes Getränkt erfasst. Der Konsum wirkt sich auf Empfehlungen für den Benutzer aus.
\end{itemize}


\subsection*{Main Success Scenario}

\begin{enumerate}
\item Der Benutzer wählt die Option Konsum erfassen
\item Der Benutzer wählt ein Bier aus einer Liste aus.
\item Der Benutzer wählt die Art (Menge) des Konsums
\item Der Benutzer speichert den Konsum.
\end{enumerate}


\subsection*{Extensions}

\begin{description}
\item[1-2a] Der Benutzer sieht eine Empfehlung an.
	\begin{enumerate}
	\item Der Benutzer kann direkt auf die Empfehlung einen Konsum erfassen.
	\end{enumerate}

\end{description}


\subsection*{Special Requirements}

\begin{itemize}
\item Die Erfassung eines Konsum soll aus jedem "Bier-Kontext" möglich sein. Bier ansehen, Bier bewerten...
\end{itemize}



\subsection*{Frequency of Occurrence}

Das Erfassen von konsumierten Getränken ist eine Kernfunktionalität welche aber nicht verwendet werden muss um Empfehlungen zu erhalten. Die Benutzungshäufigkeit wird daher auf 50% geschätzt.

\subsection*{Open Issues}

\begin{itemize}
\item nichts
\end{itemize}

\end{document}