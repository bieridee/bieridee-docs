\documentclass[10pt,a4paper]{scrartcl}
\pagestyle{empty}
\usepackage{a4} % alternativ \usepackage{a4wide}
\usepackage[ngerman]{babel} % Neudeutsche Silbentrennung (mehrsprachiges Dokument)
\usepackage{parskip} % Skip indentation of first row
\usepackage{graphicx} % Graphics support
\usepackage{longtable} % Tables across several pages
\usepackage{hyperref} % Hyperlinks
\usepackage[automark]{scrpage2} %kopf/fusszeile
\usepackage[utf8x]{inputenc} % Unicode-Encoding
 
\linespread{1.3}

\author{Danilo Bargen, Christian Fässler, Jonas Furrer} 
\title{Usecases Projekt BierIdee}
\subtitle{Brief format}

\pagestyle{scrheadings}
\ihead{SE2 Projekte} %linke Kopfzeile
\ohead{BierIdee} %rechte Kopfzeile

\begin{document}

\begin{titlepage}
	\maketitle
	\vspace{120mm}
	\thispagestyle{empty} % Don't start page numbers on this page
\end{titlepage}



\section{Usecases Projekt BierIdee}

\subsection*{UC01: User registriert sich}
Ein User kann sich registrieren um einen Benutzerzugang (Benutzername,Passwort) zu erhalten. Der Benutzer erhält eine Bestätigungs-email der Registration.
Diese Registration muss er mittels Link bestätigen und kann sich danach einloggen.
\subsection*{UC02: User loggt sich ein}
Ein User kann sich mittels Benutzernamen und Passwortkombination am System anmelden.
Der Benutzer hat die Möglichkeit seine Anmeldeinformationen zu speichern, sodass er Sie nicht jedes Mal erneut eingeben muss
\subsection*{UC03: User erfasst ein Bier}
Ein User kann ein neues Bier mit den definierten Attributen in das System einpflegen.
\subsection*{UC04: User bearbeitet erfasstes Bier}
Ein Benutzer kann die Attribute eines im System erfassten Bieres anpassen und abspeichern.
\subsection*{UC05: User bewertet Bier}
Der Benutzer kann ein Bier auswählen und jenes mit einer persönlichen Bewertung versehen. Wenn der Benutzer bereits eine Bewertung erfasst hat, wird diese angezeigt und kann angepasst werden.
\subsection*{UC06: User taggt Bier}
Der Benutzer kann ein Bier auswählen und mit freitext "taggen". Freitext Tags werden mit Autocompletion unterstützt.
\subsection*{uc07: User ruft Bier-Empfehlungen ab}
Der Benutzer kann auf sein Profil bezogene Empfehlungen für ihm bisher unbekannte (unbewertet) Biersorten abrufen.
\subsection*{UC08: User ruft Bier-Katalog ab}
Der Benutzer kann den ganzen Katalog an im System erfassten Bieren in listenartiger Übersicht durchsehen. Durch auswählen eines einzelnen Bieres kann er dessen weitere Attribute abrufen.
\subsection*{UC09: User ruft Brauerei-Katalog ab}
Der Benutzer kann den ganzen Katalog an im System erfassten Brauereien durchsehen.
\subsection*{UC10: User erfasst oder trackt Aktivität}
Der Benutzer kann eine Konsum Aktivität erfassen. 
\subsection*{UC11: User betrachet und verfolgt Timeline}
Der Benutzer kann eine Timeline betrachten. Die Timeline zeigt letzte Aktivitäten anderer Benutzer.
\subsection*{UC12: User schaut sein eigenes Profil (User-Aktivitäten) an}
Der Benutzer kann im System seine erfassten Aktivitäten (Bewertungen, Konsum) anzeigen.
\subsection*{UC13: User schaut Profile (User-Aktiviäten) eines anderen Users an}
Der Benutzer kann Profile anderer Benutzer ansehen. Er sieht dadurch Aktivitäten anderer Benutzer. Aktivitäten sind Bewertungen und Konsumationen.

\end{document}